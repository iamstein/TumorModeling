\section{Rough outline}

In this paper we will:
\begin{itemize}
\item derive AFIRT
	\begin{itemize}
		\item new result = AFIRT formula
		\item new result = Kssd
	\end{itemize}
\item use sensitivity analysis to confirm AFIRT formula and show when it fails
	\begin{itemize}
		\item large receptor concentration (herceptin = trastuzumab)
	\end{itemize}
\item specific practical results from AFIRT and sensitivity analysis
	\begin{itemize}
		\item demonstrate the importance of understanding tissue-accumulation (bevacizumab)
		\item explore how fast shed rate needs to be to limit inhibition
	\end{itemize}
\item discussion
	\begin{itemize}
		\item practical value - don't need every microconstant, just a few lumped parameter estimes
		\item don't need a complex physiological model, the simple formula can work well
		\item essentially, Kssd should have been used for atezo instead of Kd
	\end{itemize}
\end{itemize}


\section{Introduction}
During biologic drug development, prediction of target engagement at the site of action plays a critical role in dose regimen selection \cite{wang16}.  Because target engagement measurements at the site of action are often impossible to obtain, model based predictions of target engagement at the site of action are often used to help justify the dose regimen selection.  The methods used to predict target engagement vary significantly in their level of complexity and assumptions and we briefly describe here both a simple and complex approach for two different immunotherapies: pembrolizumab and atezolizumab.

For the PD-1 inhibitor pembrolizumab, a physiologically based model for antibody distribution and target engagement was developed to predict the dose needed to achieve target engagement and tumor suppression \cite{lindauer17}.  This model made many detailed assumptions about many parameters in mouse and in how these parameters would scale to humans.  For the PD-L1 inhibitor atezolizumab, a much simpler approach was taken to help justify the dose regimen, where a particular tumor biodistribution coefficient ($B$) and in vivo binding affinity ($\Kd$) was assumed, the steady state trough concentration ($\Cmin$) was estimated from clinical observations, and then the receptor occupancy ($RO$) formula in Equation~\ref{eq:ro} was used to identify the dosing regimen that would provide the drug concentration needed to achieve 95\% target occupancy.  Fewer assumptions about specific parameters are made with this simple model, but in choosing this simple model, many implicit assumptions are made, which were not all clearly stated.
\begin{equation}
RO = B\cdot\Cmin/(B\cdot\Cmin+\Kd)
\label{eq:ro}
\end{equation}

The advantage of using the complex, mechanistic model as done for pembrolizumab is that ideally, it captures all essential underlying physiology processes in making the dose regimen prediction.  The disadvantage of this approach is that physiological models can be complex, making them time consuming to develop, difficult to estimate all model parameters, and challenging to explain to collaborators.  The advantage using the simple $RO$ formula as done for atezolizumab is that it is fast, easy to implement, and easy to explain to collaborators.  The disadvantage is that it is not immediately obvious that Equation~\ref{eq:ro} is the appropriate equation to use to describe the clinical scenario as this equation was derived for the in vitro setting for chemical species\cite{boeynaems80} where the was no drug distribution, target distribution, receptor synthesis, or receptor internalization/shedding.

In this paper, a mathematical analysis of a physiologically-based model for drug disrtibution and target turnover  is performed and a simple expression for predicting target engagement in the clinical scenario is derived.  All assumptions made in deriving this formula are explicitly stated.  This paper extends previous work that focused on target engagement in circulation, as characterized by the Average Free target to Initial target Ratio (AFIR) at steady state in circulation \cite{stein17}.

PROVIDE AN OUTLINE OF THE PAPER HERE