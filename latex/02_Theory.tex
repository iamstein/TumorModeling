\newcommand{\shedM}{\text{shedM}}
\newcommand{\shedDM}{\text{shedDM}}
\newcommand{\synM}{\text{synM}}
\newcommand{\synDM}{\text{synDM}}
\newcommand{\synS}{\text{synS}} 
\newcommand{\eM}{\text{e}M}  
\newcommand{\eS}{\text{e}S}  
\newcommand{\eDM}{\text{e}DM}   
\newcommand{\eDS}{\text{e}DS}
\newcommand{\totss}{\text{tot,ss}}
\newcommand{\Det}{(k_{\text{shed}M1} + k_{13M} + k_{\eM1})(k_{\text{shed}M3} + k_{31M} + k_{\eM3}) - k_{13M}k_{31M}}
\newcommand{\DetSS}{(k_{\text{shed}DM1} + k_{13DM} + k_{\eDM1})(k_{\text{shed}DM3} + k_{31DM} + k_{\eDM3})}

\section{Theory}

In this section, an expression for the average free target to initial target ratio in tissue (AFIRT) is derived for the model in Figure~\ref{fig:model}.

\subsection{Model Description}
The model in Figure~\ref{fig:model} is based on the standard TMDD model \cite{mager01, stein17}, where a drug ($D$) binds its target, but it has been extended to include the following processes:
\begin{itemize}
\item Shedding of the membrane bound target ($M$) into soluble target ($S$).  
\item The drug can bind both the soluble target and the membrane-bound target to form complexes $DS$ and $DM$ respectively.
\item Both the soluble and membrane-bound target are present in both the central compartment (1) and a tissue compartment (3).
\item Soluble target is able to distribute between the central and tissue compartment (e.g. passive diffusion of target molecules).
\item Membrane-bound target is able to  distribute between central and tissue compartment (e.g. active trafficking of immune cells expressing the membrane-bound target).
\end{itemize}
As with the standard TMDD model, the peripheral compartment (2) contains the drug only; it is included so that the two-compartment pharmacokinetics of the drug can be described.  This model is more complex that most physiological models in the literature.  Often, trafficking of the membrane-bound target is ignored and often, either the soluble or membrane-bound target is modeled but not both.  In this paper, an analysis of this complete model is performed.  The simpler scenarios can easily be represented using this model and setting the appropriate parameters to zero.  

The ordinary differential equations for describing this system are summarized below.

\begin{align}
\frac{dD_1}{dt} &= \frac{1}{V_C}\text{Dose}_{iv}(t) - k_{12D}D_1 + \frac{V_{D_2}}{V_{D_1}}k_{21D}D_2 - k_{13D}D_1 \nonumber 
+ \frac{V_{D_3}}{V_{D_1}}k_{31D}D_3\\& - k_{eD1}D_1 - k_{\text{on}1}D_1 \cdot S_1 + k_{\text{off}1}(DS_1)- k_{\text{on}1}D_1 \cdot M_1 + k_{\text{off}1}(DM_1)  \\
\frac{dD_2}{dt} &= k_{12D}\frac{V_{D_1}}{V_{D_2}}D_1 - k_{21D}D_2 \\
\frac{dD_3}{dt} &= \frac{V_{D_1}}{V_{D_3}}k_{13D}D_1 - k_{31D}D_3 - k_{eD3}D_3- k_{\text{on}3}D_3 \cdot M_3 + k_{\text{off}3}(DM_3)\nonumber  \\&- k_{\text{on}3}D_3 \cdot S_3 + k_{\text{off}3}(DS_3) \\
\frac{dM_1}{dt} &= k_{\text{syn}M1}-k_{\text{shed}M_1}M_1 - k_{13M}M_1 + \frac{V_{M_3}}{V_{M_1}}k_{31M}M_3- k_{eM1}M_1\nonumber  \\&- k_{\text{on}1}D_1 \cdot M_1 + k_{\text{off}1}(DM_1) \\
\frac{dM_3}{dt} &= k_{\text{syn}M3}-k_{\text{shed}M_3}M_3  + \frac{V_{M_1}}{V_{M_3}}k_{13M}M_1 - k_{31M}M_3 - k_{eM3}M_3\nonumber \\&- k_{\text{on}3}D_3 \cdot M_3 + k_{\text{off}3}(DM_3) \\
\frac{dS_1}{dt} &= k_{\text{syn}S1}+k_{\text{shed}M_1}M_1  - k_{13S}S_1 + \frac{V_{S_3}}{V_{S_1}}k_{31S}S_3 - k_{eS1}S_1 \nonumber\\ &- k_{\text{on}1}D_1 \cdot S_1  + k_{\text{off}1}(DS_1)  \\
\frac{dS_3}{dt} &= k_{\text{syn}S3}+k_{\text{shed}M_3}M_3 + \frac{V_{S_1}}{V_{S_3}}k_{13S}S_1 - k_{31S}S_3 - k_{eS3}S_3 \nonumber\\ &- k_{\text{on}3}D_3 \cdot S_3 + k_{\text{off}3}(DS_3)  \\
\frac{d(DM_1)}{dt} &=-k_{\text{shed}DM_1}DM_1 - k_{13DM}(DM_1) + \frac{V_{DM_3}}{V_{DM_1}}k_{31DM}(DM_3)  - k_{eDM1}DM_1\nonumber\\&+k_{\text{on}1}D_1 \cdot M_1 - k_{\text{off}1}(DM_1)  \\
\frac{d(DM_3)}{dt} &= -k_{\text{shed}DM_3}DM_3 + \frac{V_{DM_1}}{V_{DM_3}}k_{13DM}(DM_1) - k_{31DM}(DM_3) - k_{eDM3}DM_3\nonumber \\&+k_{\text{on}3}D_3 \cdot M_3 - k_{\text{off}3}(DM_3)  \\
\frac{d(DS_1)}{dt} &=k_{\text{shed}DM_1}DM_1  - k_{13DS}DS_1 + \frac{V_{DS_3}}{V_{DS_1}}k_{31DS}DS_3 - k_{eDS1}(DS_1)  \nonumber\\ &+ k_{\text{on}1}D_1 \cdot S_1 - k_{\text{off}1}(DS_1)  \\
\frac{d(DS_3)}{dt} &=k_{\text{shed}DM_3}DM_3 + \frac{V_{DS_1}}{V_{DS_3}}k_{13DS}DS_1 - k_{31DS}DS_3 - k_{eDS3}(DS_3)  \nonumber\\ &+ k_{\text{on}3}D_1 \cdot S_3 - k_{\text{off}1}(DS_3)  
\end{align}

For the first time we are calculating the AFIRT using an alternative approximations to the quasi-equilibrium (QE), the quasi-steady state (QSS) and the quasi-steady state with distribution (QSSD). The quasi-steady state approximation assumed that $-k_{\text{shed}DM_3}DM_3 - k_{eDM3}DM_3+k_{\text{on}3}D_3 \cdot M_3 - k_{\text{off}3}(DM_3) = 0 $, similarly as before for $k_d$ we define $k_{ss}=\frac{k_{\text{shed}DM_3} + k_{eDM3}+ k_{\text{off}3}}{k_{\text{on}3}}$. 
The quasi-steady state with distribution approximation assumed that $-k_{\text{shed}DM_3}DM_3-k_{31DM}(DM_3) - k_{eDM3}DM_3+k_{\text{on}3}D_3 \cdot M_3 - k_{\text{off}3}(DM_3) = 0 $, we define $k_{ssd}=\frac{k_{\text{shed}DM_3} +k_{31DM}+ k_{eDM3}+ k_{\text{off}3}}{k_{\text{on}3}}$. 
These approximations are obtained from equation (16) at steady state, i.e. $\frac{d(DM_3)}{dt}=0$, and assuming the terms $ \frac{V_{DM_1}}{V_{DM_3}}k_{13DM}(DM_1)$ and $ k_{31DM}(DM_3) $ are negligible for the QSS. We assume that only $\frac{V_{DM_1}}{V_{DM_3}}k_{13DM}(DM_1)$ is negligible for the QSSD.

DESCRIBE THE INITIAL CONDITIONS OF THE MODEL



\subsection{AFIRT and TRIFT derivation}




The general formula that will be derived here is:
\[\tag{3.2.1}
    \frac{T_3}{T_{30}}  = (\frac{T_3}{T_{\text{tot,ss}}})
                        (\frac{T_{\text{tot,ss}}}{T_{30}})
\] 


Depending on particular drug, the target can be membrane-bound or soluble.
Free Fraction Coefficient(FFC) measures how fast drug-target complex unbinds
\[\tag{3.2.2}
    \Keq = \frac{D_3\cdot T_3}{DT_3}
\]
This gives
\[\tag{3.2.3}
    DT_3\times\Keq = D_3\cdot T_3
\]
Substitute $DT_3$ for $T_{\text{tot},3} - T_3$ in (3.2.3) and solve
for $T_3/T_{\text{tot},3}$, one has
\[\tag{3.2.4}
    \frac{T_3}{T_{\text{tot},3}} = \frac{\Keq}{\Keq + D_3} 
    \cong \frac{\Keq}{D_3} \cong \frac{\Keq}{D_{\text{tot}}}
\]
The first approximation holds when $D_3 \gg \Keq$ and the second
approximation holds when $D_{\text{tot}} \cong D$, which occurs
when drug is dosed in vast molar excess to the target, as is in 
the case for most mAb drugs in clinics.

The second ratio of equation (3.2.1) is called 
Target Accumulation Rate $T_{\text{acc}}$,
\[\tag{3.2.5}
    T_{\text{acc}} = \frac{T_{\text{tot,3}}}{T_{30}}
\]
Combining (3.2.4) and (3.2.5), one has
\[
    \frac{T_3}{T_{30}} \cong 
    \frac{\Keq\cdot T_{\text{acc}}}{D_{\text{tot,3}}}
\]

The drug concentration $D_3 \cong D_{\text{tot,3}}$ is usually denoted
by $C(t)$ in literature. As mention in [CITE PREVIOUS AFIR PAPER],
the Average Free target concentration to Intial target concentration 
ratio in Tissue (AFIRT) can be computed as
\[
    \text{AFIRT} = 
    \frac{1}{\tau}\int_{t_{\text{ss}}}^{t_{\text{ss}} + \tau}
    \frac{\Keq\cdot T_{\text{acc}}}{C(t)}
\]

The Trough Free target concentration to Initial target concentration
Ratio in Tissue (TFIRT) can be computed as
\[
    \text{TFIRT} = 
    \frac{\Keq\cdot T_{\text{acc}}}{C_{\min}}
\]

With the assumption of constant dosing, $\text{AFIRT} = \text{TFIRT}$.
This is because for constant dosing, the drug concentration $C(t)$ becomes
constant during steady state. Throughout the rest of the paper, 
we assume constant dosing.




The calculation of FFC will depend on the approximations to 
quasi-equilibrium. As mentioned before this section, we will use
quasi-steady state and the quasi-steady state with distribution as
approximations.

To calculate FFC for membrane-bound target, one sets 
\[
    \frac{d(DM_3)}{dt} = 0
\]
as drug-target complexes stabilize in the steady-state.
Then using equation 10, and assuming $k_{13DM}$ is negligible, one can
compute FFC.
For membrane-bound target, the following three FFC will be used for AFIRT
\begin{itemize}
    \item For quasi-steady state with distribution
    \[
        \Kssd  = \frac{\koff + \keDM + \kshedDM +k_{\text{31DM}}}{\kon}.
    \]
    \item For quasi-steady state
    \[
        \Kss   = \frac{\koff + \keDM + \kshedDM}{\kon}.
    \]
    \item Assuming unbinding is much faster than shedding and elimination
    \[
        \Kd    = \frac{\koff}{\kon}.
    \]
\end{itemize}

Likewise, the following three FFC will be used for soluable target
depending on the initial assumption
\begin{align*}
    \Kssd & = \frac{\koff + \keDS + k_{\text{31DS}}}{\kon}, \\
    \Kss  & = \frac{\koff + \keDS}{\kon}, \\
    \Kd   & = \frac{\koff}{\kon}.
\end{align*}
Their derivation can be found in the appendix.

For membrane-bound target, the target accumulation rate can be computed as
\[
    \Tacc = \frac{T_{3\text{tot,ss}}}{T_{30}}
\]
where $T_{\text{tot,ss}}$ means drug-target complex concentration at steady state, and 
$T_{30}$ means target concentration at the initial state.
At the initial state, the change of target in $D_1$ and $D_3$ are both zero. 
Therefore, to solve $T_{10}$ and $T_{30}$, one can set equation 5 and 6 
(resp, 7 and 8 for soluable target) to be zero
and solve the associated linear system. 
At the steady state, targets in both $D_1$ and $D_3$ are 
in the form of complexes, and the change of complex concentration is zero. 
Therefore, to solve $T_{1\text{tot,ss}}$ and $T_{3\text{tot,ss}}$, one can set equation 9 and 10 (resp, 11 and 12 for soluble target)
to be zero, and solve the linear system. Their derivation can be found in Section \ref{initial and steady state values}.\\

\begin{align}
M_{10} & = \frac{(k_{\text{shed}M_3} + k_{31M} + k_{\e M3})k_{\synM1} + (V_{T}/V_{C})k_{31M}k_{\synM3}}{(k_{\text{shed}M_1} + k_{13M} + k_{\e M1})(k_{\text{shed}M_3} + k_{31M} + k_{\e M3}) - k_{31M}k_{13M}}\\
M_{30} & = \frac{(V_{C}/V_{T})k_{13M}k_{\synM1} + (k_{\text{shed}M_1} + k_{13M} + k_{\e M1})k_{\synM3}}{(k_{\text{shed}M_1} + k_{13M} + k_{\e M1})(k_{\text{shed}M_3} + k_{31M} + k_{\e M3}) - k_{31M}k_{13M}}\\
S_{10} & = \frac{(k_{31S} + k_{\e S3})(k_{\synS1} + k_{\shedM_1}M_{10}) + (V_{T}/V_{C})k_{31S}(k_{\synS3} + k_{\shedM_3}M_{30})}{(k_{13S} + k_{\e S1})(k_{31S} + k_{\e S3}) - k_{31S}k_{13S}}\\
S_{30} & = \frac{(V_{C}/V_{T})k_{13S}(k_{\synS1} + k_{\shedM_1}M_{10}) + (k_{13S} + k_{\e S1})(k_{\synS3} + k_{\shedM_3}M_{30})}{(k_{13S} + k_{\e S1})(k_{31S} + k_{\e S3}) - k_{31S}k_{13S}}\\
M_{1\totss} & = \frac{(k_{\text{shed}DM_3} + k_{31DM} + k_{\e DM3})k_{\synM1} + (V_{T}/V_{C})k_{31DM}k_{\synM3}}   {(k_{\text{shed}DM_1} + k_{13DM} + k_{\e DM1})(k_{\text{shed}DM_3} + k_{31DM} + k_{\e DM3}) - k_{31DM}k_{13DM}}\\
M_{3\totss} & = \frac{(V_{C}/V_{T})k_{13DM}k_{\synM1} + (k_{\text{shed}DM_1} + k_{13DM} + k_{\e DM1})k_{\synM3}}   {(k_{\text{shed}DM_1} + k_{13DM} + k_{\e DM1})(k_{\text{shed}DM_3} + k_{31DM} + k_{\e DM3}) - k_{31DM}k_{13DM}}\\
S_{1\totss} & = \frac{(k_{31DS} + k_{\e DS3})(k_{\synS1} + k_{\shedDM_1}DM_{1\totss}) + (V_{T}/V_{C})k_{31DS}(k_{\synS3} + k_{\shedDM_3}DM_{3\totss})}   {(k_{13DS} + k_{\e DS1})(k_{31DS} + k_{\e DS3}) - k_{31DS}k_{13DS}}\\
S_{3\totss} & = \frac{(V_{C}/V_{T})k_{13DS}(k_{\synS1} + k_{\shedDM_1}DM_{1\totss}) + (k_{13DS} + k_{\e DS1})(k_{\synS3} + k_{\shedDM_3}DM_{3\totss})}   {(k_{13DS} + k_{\e DS1})(k_{31DS} + k_{\e DS3}) - k_{31DS}k_{13DS}}
\end{align}


I PROPOSE THAT THIS SECTION START WITH A HIGH LEVEL OVERVIEW OF THE STEPS TO CALCULATE AFIRT.  THEN, THESE STEPS CAN EITHER BE REPRODUCED IN THE MAIN BODY OR IN THE APPENDIX.  

KEY STEPS
\begin{itemize}
\item Focus on steady state for a constant infusion.
\item Assume the drug concentration is very large
\item Calculate drug concentration in tissue at steady state $D_\text{tot,avg3,ss} = C_\text{avg3,ss} = B\cdot C_\text{avg1,ss} = D_\text{tot,avg1,ss}.$
	\begin{itemize}
		\item  For large drug concentration, target kinetics can be ignored. 
		\item Use the similarity transform to remove $k_\text{eD3}$.
		\item This gives us a simple expression for $C_\text{avg1} = D_\text{tot,avg1}$ as a function of $\CL$.
		\item This also gives us an expression for $B$ such that $C_\text{avg3} = D_\text{tot,avg3} = B\cdot C_\text{avg1}$.
	\end{itemize}
\item Calculate target concentration in tissue at steady state.
	\begin{itemize}
		\item For large drug concentration, drug kinetics can be ignored.
		\item Do the small matrix algebra inversion to calculate $M_{30}$ and $M_{3tot,ss}$.
		\item Calculate accumulation factor
	\end{itemize}
\item Calculate the free fraction by assuming various processes are slow/fast.
	\begin{itemize}
		\item $\Kd$
		\item $\Kss$
		\item $\Kssd$
	\end{itemize}	
\item Putting this all together gives $AFIRT_M$.  
\item A similar process can be followed for $AFIRT_S$.	
\end{itemize}



\subsubsection{Membrane-bound Target}

 

Formulas for computing membrane bound target at initial and steady state can be found in Section \ref{initial and steady state values}.

We obtain the following approximations 
\begin{align*}
\text{AFIRT} \approx \Kssd\frac{M_{3\text{tot,ss}}}{M_{3,0}}\frac{\text{CL}\times\tau}{\text{B}\times\text{Dose}},\\
\text{AFIRT} \approx \Kss\frac{M_{3\text{tot,ss}}}{M_{3,0}}\frac{\text{CL}\times\tau}{\text{B}\times\text{Dose}},\\
\text{AFIRT} \approx \Kd\frac{M_{3\text{tot,ss}}}{M_{3,0}}\frac{\text{CL}\times\tau}{\text{B}\times\text{Dose}}.
\end{align*}  

We can notice from the picture below that the approximation obtained using the (QSS) hypothesis gives better results than the one using (QE).



\subsubsection{Soluble target at Initial State and Steady State}



Formulas for computing soluble at initial and steady state can be found in Section \ref{initial and steady state values}.

The formulas to compute various AFIRTS for soluble targets in the tumor compartment are 
\begin{align}
    \text{AFIRTS}\Kssd & = K_{\text{ssd.S}}\cdot \frac{T_{\text{acc.tum.S}}}{\text{B}\cdot C_{\text{avg}1}} \\
   \text{AFIRTS}\Kss & = K_{\text{ss.S}}\cdot \frac{T_{\text{acc.tum.S}}}{\text{B}\cdot C_{\text{avg}1}} \\
     \text{AFIRTS}\Kd & = K_{\text{d.S}}\cdot \frac{T_{\text{acc.tum.S}}}{\text{B}\cdot C_{\text{avg}1}} \\
\end{align}
Where $K_{\text{ssd.S}}$, $K_{\text{ss.S}}$ and $K_{\text{d.S}}$
\begin{align}
    K_{\text{ssd.S}} & = \frac{k_{\shedDM3} + k_{31DS} + k_{\eDS} + k_{\off3}}{k_{\on3}} \\
    \text{Kss.S}  & = \frac{k_{\shedDM3} + k_{\eDS} + k_{\off3}}{k_{\on3}} \\
    \text{Kd.S}   & = \frac{k_{\off3}}{k_{\on3}} \\
\end{align}
 Finally,
\begin{align}
    T_{\text{acc.tum.S}} = S_{\text{3tot.ss}} / S_{30}.
\end{align}


To compute AFIRT for soluable target from simulation, do
\begin{enumerate}
\item compute $\text{Sfree.pct}:=\frac{S_3}{S_{30}}$
\item take the average of Sfree.pct in the steady steady state
\item return the result from step 2
\end{enumerate}
