\section{Theory}

\subsection{Model Equations}
The drug and target dynamics are modeled with the following system of ordinary differential equations.  The model is shown in Figure~\ref{fig:model}.  THOUGH THIS FIGURE PROBABLY NEEDS TO BE UPDATED.
\begin{align}
\frac{dD_1}{dt} &= \frac{1}{V_C}\text{Dose}_{iv}(t) - k_{12D}D_1 + \frac{V_{D_2}}{V_{D_1}}k_{21D}D_2 - k_{13D}D_1 \nonumber 
+ \frac{V_{D_3}}{V_{D_1}}k_{31D}D_3\\& - k_{eD1}D_1 - k_{\text{on}1}D_1 \cdot S_1 + k_{\text{off}1}(DS_1)- k_{\text{on}1}D_1 \cdot M_1 + k_{\text{off}1}(DM_1)  \\
\frac{dD_2}{dt} &= k_{12D}\frac{V_{D_1}}{V_{D_2}}D_1 - k_{21D}D_2 \\
\frac{dD_3}{dt} &= \frac{V_{D_1}}{V_{D_3}}k_{13D}D_1 - k_{31D}D_3 - k_{eD3}D_3- k_{\text{on}3}D_3 \cdot M_3 + k_{\text{off}3}(DM_3)\nonumber  \\&- k_{\text{on}3}D_3 \cdot S_3 + k_{\text{off}3}(DS_3) \\
\frac{dM_1}{dt} &= k_{\text{syn}M1}-k_{\text{shed}M_1}M_1 - k_{13M}M_1 + \frac{V_{M_3}}{V_{M_1}}k_{31M}M_3- k_{eM1}M_1\nonumber  \\&- k_{\text{on}1}D_1 \cdot M_1 + k_{\text{off}1}(DM_1)  \\
\frac{dM_3}{dt} &= k_{\text{syn}3}-k_{\text{shed}M_3}M_3  + \frac{V_{M_1}}{V_{M_3}}k_{13M}M_1 - k_{31M}M_3 - k_{eM3}M_3-\nonumber \\& k_{\text{on}3}D_3 \cdot M_3 + k_{\text{off}3}(DM_3) \\
\frac{dS_1}{dt} &= k_{\text{syn}S1}+k_{\text{shed}M_1}M_1  - k_{13S}S_1 + \frac{V_{S_3}}{V_{S_1}}k_{31S}S_3 - k_{eS1}S_1 - k_{\text{on}1}D_1 \cdot S_1 \nonumber\\ & + k_{\text{off}1}(DS_1)  \\
\frac{dS_3}{dt} &= k_{\text{syn}S3}+k_{\text{shed}M_3}M_3  - k_{31S}S_3 + \frac{V_{S_1}}{V_{S_3}}k_{13S}S_1 - k_{eS3}S_3 - k_{\text{on}3}D_3 \cdot S_3 \nonumber\\ & + k_{\text{off}3}(DS_3)  \\
\frac{d(DM_1)}{dt} &=-k_{\text{shed}DM_1}DM_1 - k_{13DM}(DM_1) + \frac{V_{DM_3}}{V_{DM_1}}k_{31DM}(DM_3)  - k_{eDM1}DM_1\nonumber\\&+k_{\text{on}1}D_1 \cdot M_1 - k_{\text{off}1}(DM_1)  \\
\frac{d(DM_3)}{dt} &= -k_{\text{shed}DM_3}DM_3 - k_{31DM}(DM_3) + \frac{V_{DM_1}}{V_{DM_3}}k_{13DM}(DM_1)  - k_{eDM3}DM_3\nonumber \\&+k_{\text{on}3}D_3 \cdot M_3 - k_{\text{off}3}(DM_3)  \\
\frac{d(DS_1)}{dt} &=-k_{\text{shed}DM_3}DS_1  - k_{13DS}DS_1 + \frac{V_{DS_3}}{V_{DS_1}}k_{31DS}DS_3 - k_{eDS1}(DS_1)  \nonumber\\ &+ k_{\text{on}1}D_1 \cdot S_1 - k_{\text{off}1}(DS_1)  \\
\frac{d(DS_3)}{dt} &=-k_{\text{shed}DM_3}DS_3  - k_{31DS}DS_3 + \frac{V_{DS_1}}{V_{DS_3}}k_{13DS}DS_1 - k_{eDS3}(DS_3)  \nonumber\\ &+ k_{\text{on}3}D_1 \cdot S_3 - k_{\text{off}1}(DS_3)  
\end{align}

This model is considering three compartments like Model C but it is taking into account the possibility for the target to shad, i.e. the target in the previous models is always membrane-bound but in reality it can shad into the blood and into the tumor. We are also adding the lymphocyte trafficking ($k_{13M}$ and $k_{31M}$).
The analysis we are carrying out on this model is similar to the one done for model C with the required modification. For the first time we are calculating the AFIRT using an alternative approximations to the quasi-equilibrium (QE), the quasi-steady state (QSS) and the quasi-steady state xxxxxxx (QSSD). The quasi-steady state approximation assumed that $-k_{\text{shed}DM_3}DM_3 - k_{eDM3}DM_3+k_{\text{on}3}D_3 \cdot M_3 - k_{\text{off}3}(DM_3) = 0 $, similarly as before for $k_d$ we define $k_{ss}=\frac{k_{\text{shed}DM_3} + k_{eDM3}+ k_{\text{off}3}}{k_{\text{on}3}}$. 

\subsection{AFIRT derivation}

\subsubsection{Membrane-bound Target}

The quasi-steady state xxxxxxx approximation assumed that $-k_{\text{shed}DM_3}DM_3-k_{31DM}(DM_3) - k_{eDM3}DM_3+k_{\text{on}3}D_3 \cdot M_3 - k_{\text{off}3}(DM_3) = 0 $, we define $k_{ssd}=\frac{k_{\text{shed}DM_3} +k_{31DM}+ k_{eDM3}+ k_{\text{off}3}}{k_{\text{on}3}}$. 
These approximations are obtained from equation (16) at steady state, i.e. $\frac{d(DM_3)}{dt}=0$, and assuming the terms $ \frac{V_{DM_1}}{V_{DM_3}}k_{13DM}(DM_1)$ and $ k_{31DM}(DM_3) $ are negligible for the QSS. We assume that only $\frac{V_{DM_1}}{V_{DM_3}}k_{13DM}(DM_1)$ is negligible for the QSSD. 

We obtain the following approximations 
\begin{align*}
\text{AFIRT} \approx k_{ssd}\frac{M_{3\text{tot,ss}}}{M_{3,0}}\frac{\text{CL}\times\tau}{\text{B}\times\text{Dose}},\\
\text{AFIRT} \approx k_{ss}\frac{M_{3\text{tot,ss}}}{M_{3,0}}\frac{\text{CL}\times\tau}{\text{B}\times\text{Dose}},\\
\text{AFIRT} \approx k_{d}\frac{M_{3\text{tot,ss}}}{M_{3,0}}\frac{\text{CL}\times\tau}{\text{B}\times\text{Dose}},
\end{align*}  
where 
\begin{align*}
M_{3\text{tot,ss}}&=\frac{k_{13DM}(V_C/V_T)k_{\text{syn}M1}+(k_{eDM1}+k_{\text{shed}DM_1}+k_{13DM})k_{\text{syn}M3}}{(k_{eDM1}+k_{\text{shed}DM_1}+k_{13DM})(k_{eDM3}+k_{\text{shed}DM_3}+k_{31DM})-k_{31DM}k_{13DM}},\\
M_{3,0}&=\frac{k_{13M}(V_C/V_T)k_{\text{syn}M1}+(k_{eM1}+k_{\text{shed}M_1}+k_{13M})k_{\text{syn}M3}}{(k_{eM1}+k_{\text{shed}M_1}+k_{13M})(k_{eM3}+k_{\text{shed}M_3}+k_{31M})-k_{31M}k_{13M}},\\
\text{B}&=\frac{k_{13D}(V_C/V_T)}{k_{eD3}+k_{31D}}.
\end{align*}
We can notice from the picture below that the approximation obtained using the (QSS) hypothesis gives better results than the one using (QE).

% ---------------------------------------------------------------
% Soluble Target 
% ---------------------------------------------------------------
\subsubsection{Soluble target at Initial State and Steady State}
With the same notations as in the sections above.\newline 
\textbf{At initial state}, we have
\begin{align*}
    \frac{dS_1}{dt} & = 0 \\
    \frac{dS_3}{dt} & = 0 \\
                M_1 & = M_{10} \\
                M_3 & = M_{30} \\
                D_1 & = 0 \\
                D_3 & = 0 \\
               DS_1 & = 0 \\
               DS_3 & = 0 
\end{align*}
So ODE 13 and 14 give us the following linear system
\[
    \matrix{-(k_{13S} + k_{eS1})}{\frac{V_T}{V_C}k_{31S}}{\frac{V_C}{V_T}k_{13S}}{-(k_{31S} + k_{eS3})}
        \cdot \vector{S_1}{S_3}_0 = \vector{-k_{synS1} - k_{shedM1}M_{10}}{-k_{synS3} - k_{shedM3}M_{30}}.
\]
So the initial soluble target concentration in the central and tumor compartment is
\begin{align}
    S_{10} & = \frac{(k_{31S} + k_{eS3})\cdot(k_{synS1} + k_{shedM1}M_{10})
                    + \frac{V_T}{V_C}k_{31S}(k_{synS3} + k_{shedM3}M_{30})}
                {(k_{13S} + k_{eS1})(k_{31S} + k_{eS3}) - k_{13S}k_{31S}} \\
    S_{30} & = \frac{\frac{V_C}{V_T}\cdot k_{13S}\cdot(k_{synS1} + k_{shedM1}M_{10})
                    + (k_{13S} + k_{eS1})(k_{synS3} + k_{shedM3}M_{30})}
            {(k_{13S} + k_{eS1})(k_{31S} + k_{eS3}) - k_{13S}k_{31S}}.
\end{align}

\textbf{At steady state}, we assume that all soluble target are in the form of bounded complex.
By the symmetry of our model, we have
\begin{align*}
    S_{1tot,ss} & = \frac{(k_{31DS} + k_{eDS3})\cdot(k_{synS1} + k_{shedDM1}M_{1tot,ss})
                    + \frac{V_T}{V_C}k_{31DS}(k_{synS3} + k_{shedDM3}M_{3tot,ss})}
                {(k_{13DS} + k_{eDS1})(k_{31DS} + k_{eDS3}) - k_{13DS}k_{31DS}} \\
    S_{3tot,ss} & = \frac{\frac{V_C}{V_T} k_{13DS}\cdot(k_{synS1} + k_{shedDM1}M_{1tot,ss})
                    + (k_{13DS} + k_{eDS1})(k_{synS3} + k_{shedDM3}M_{3tot3,ss})}
            {(k_{13DS} + k_{eDS1})(k_{31DS} + k_{eDS3}) - k_{13DS}k_{31DS}}
\end{align*}

Formulas for computing membrane target at initial and steady state can be found in 
\emph{Model F Appendix}, for convenience, we include them below

\newcommand{\Det}{(k_{shedM1} + k_{13M} + k_{eM1})(k_{shedM3} + k_{31M} + k_{eM3}) - k_{13M}k_{31M}}

\newcommand{\DetSS}{(k_{shedDM1} + k_{13DM} + k_{eDM1})(k_{shedDM3} + k_{31DM} + k_{eDM3})
- k_{13DM}k_{31DM}}
\begin{align*}
    M_{10} & = \frac{(k_{shedM3} + k_{31M} + k_{eM3})k_{synM1} + \frac{V_T}{V_C}k_{31M}k_{synM3}}
                    {\Det} \\
    M_{30} & = \frac{\frac{V_C}{V_T}k_{13M}k_{synM1} + (k_{shedM1} + k_{13M} + k_{eM1})k_{synM1}}
                    {\Det} \\
    M_{1tot,ss} & = \frac{(k_{shedDM3} + k_{31DM} + k_{eDM3})k_{synM1} + 
                            \frac{V_T}{V_C}k_{31DM}k_{synM3}}
                    {\DetSS} \\
    M_{3tot,ss} & = \frac{\frac{V_C}{V_T}k_{13DM}k_{synM1} + 
                            (k_{shedDM1} + k_{13DM} + k_{eDM1})k_{synM1}}
                    {\DetSS}
\end{align*}


The formulas to compute various AFIRTS for soluble targets in the tumor compartment are 
\begin{align}
    \text{AFIRTS.Kssd} & = \text{Kssd.S}\times \frac{\text{Tacc.tum.S}}{\text{B}\times C_{avg1}} \\
    \text{AFIRTS.Kss}  & = \text{Kss.S}\times \frac{\text{Tacc.tum.S}}{\text{B}\times C_{avg1}} \\
    \text{AFIRTS.Kd}   & = \text{Kd.S}\times \frac{\text{Tacc.tum.S}}{\text{B}\times C_{avg1}} \\
\end{align}
Where $\text{Kssd.S}$, $\text{Kss.S}$ and $\text{Kd.S}$
\begin{align}
    \text{Kssd.S} & = \frac{k_{shedDM3} + k_{31DS} + k_{eDS} + k_{off3}}{k_{on3}} \\
    \text{Kss.S}  & = \frac{k_{shedDM3} + k_{eDS} + k_{off3}}{k_{on3}} \\
    \text{Kd.S}   & = \frac{k_{off3}}{k_{on3}} \\
\end{align}
 Finally,
\begin{align}
    \text{Tacc.tum.S} = \text{S3tot.ss} / \text{S30}
\end{align}


To compute AFIRT for soluable target from simulation, do
\begin{enumerate}
\item compute $\text{Sfree.pct}:=\frac{S_3}{S_{30}}$
\item take the average of Sfree.pct in the steady steady state
\item return the result from step 2
\end{enumerate}