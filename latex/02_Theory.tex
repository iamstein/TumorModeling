\section*{Theory}

The drug and target dynamics are modeled with the following system of ordinary differential equations.  The model is shown in Figure~\ref{fig:model}.  THOUGH THIS FIGURE PROBABLY NEEDS TO BE UPDATED.
\begin{align}
\frac{dD_1}{dt} &= \frac{1}{V_C}\text{Dose}_{iv}(t) - k_{12D}D_1 + \frac{V_{D_2}}{V_{D_1}}k_{21D}D_2 - k_{13D}D_1 \nonumber 
+ \frac{V_{D_3}}{V_{D_1}}k_{31D}D_3\\& - k_{eD1}D_1 - k_{\text{on}1}D_1 \cdot S_1 + k_{\text{off}1}(DS_1)- k_{\text{on}1}D_1 \cdot M_1 + k_{\text{off}1}(DM_1)  \\
\frac{dD_2}{dt} &= k_{12D}\frac{V_{D_1}}{V_{D_2}}D_1 - k_{21D}D_2 \\
\frac{dD_3}{dt} &= \frac{V_{D_1}}{V_{D_3}}k_{13D}D_1 - k_{31D}D_3 - k_{eD3}D_3- k_{\text{on}3}D_3 \cdot M_3 + k_{\text{off}3}(DM_3)\nonumber  \\&- k_{\text{on}3}D_3 \cdot S_3 + k_{\text{off}3}(DS_3) \\
\frac{dM_1}{dt} &= k_{\text{syn}M1}-k_{\text{shed}M_1}M_1 - k_{13M}M_1 + \frac{V_{M_3}}{V_{M_1}}k_{31M}M_3- k_{eM1}M_1\nonumber  \\&- k_{\text{on}1}D_1 \cdot M_1 + k_{\text{off}1}(DM_1)  \\
\frac{dM_3}{dt} &= k_{\text{syn}3}-k_{\text{shed}M_3}M_3  + \frac{V_{M_1}}{V_{M_3}}k_{13M}M_1 - k_{31M}M_3 - k_{eM3}M_3-\nonumber \\& k_{\text{on}3}D_3 \cdot M_3 + k_{\text{off}3}(DM_3) \\
\frac{dS_1}{dt} &= k_{\text{syn}S1}+k_{\text{shed}M_1}M_1  - k_{13S}S_1 + \frac{V_{S_3}}{V_{S_1}}k_{31S}S_3 - k_{eS1}S_1 - k_{\text{on}1}D_1 \cdot S_1 \nonumber\\ & + k_{\text{off}1}(DS_1)  \\
\frac{dS_3}{dt} &= k_{\text{syn}S3}+k_{\text{shed}M_3}M_3  - k_{31S}S_3 + \frac{V_{S_1}}{V_{S_3}}k_{13S}S_1 - k_{eS3}S_3 - k_{\text{on}3}D_3 \cdot S_3 \nonumber\\ & + k_{\text{off}3}(DS_3)  \\
\frac{d(DM_1)}{dt} &=-k_{\text{shed}DM_1}DM_1 - k_{13DM}(DM_1) + \frac{V_{DM_3}}{V_{DM_1}}k_{31DM}(DM_3)  - k_{eDM1}DM_1\nonumber\\&+k_{\text{on}1}D_1 \cdot M_1 - k_{\text{off}1}(DM_1)  \\
\frac{d(DM_3)}{dt} &= -k_{\text{shed}DM_3}DM_3 - k_{31DM}(DM_3) + \frac{V_{DM_1}}{V_{DM_3}}k_{13DM}(DM_1)  - k_{eDM3}DM_3\nonumber \\&+k_{\text{on}3}D_3 \cdot M_3 - k_{\text{off}3}(DM_3)  \\
\frac{d(DS_1)}{dt} &=-k_{\text{shed}DM_3}DS_1  - k_{13DS}DS_1 + \frac{V_{DS_3}}{V_{DS_1}}k_{31DS}DS_3 - k_{eDS1}(DS_1)  \nonumber\\ &+ k_{\text{on}1}D_1 \cdot S_1 - k_{\text{off}1}(DS_1)  \\
\frac{d(DS_3)}{dt} &=-k_{\text{shed}DM_3}DS_3  - k_{31DS}DS_3 + \frac{V_{DS_1}}{V_{DS_3}}k_{13DS}DS_1 - k_{eDS3}(DS_3)  \nonumber\\ &+ k_{\text{on}3}D_1 \cdot S_3 - k_{\text{off}1}(DS_3)  
\end{align}

This model is considering three compartments like Model C but it is taking into account the possibility for the target to shad, i.e. the target in the previous models is always membrane-bound but in reality it can shad into the blood and into the tumor. We are also adding the lymphocyte trafficking ($k_{13M}$ and $k_{31M}$).
The analysis we are carrying out on this model is similar to the one done for model C with the required modification. For the first time we are calculating the AFIRT using an alternative approximations to the quasi-equilibrium (QE), the quasi-steady state (QSS) and the quasi-steady state xxxxxxx (QSSD). The quasi-steady state approximation assumed that $-k_{\text{shed}DM_3}DM_3 - k_{eDM3}DM_3+k_{\text{on}3}D_3 \cdot M_3 - k_{\text{off}3}(DM_3) = 0 $, similarly as before for $k_d$ we define $k_{ss}=\frac{k_{\text{shed}DM_3} + k_{eDM3}+ k_{\text{off}3}}{k_{\text{on}3}}$. 
The quasi-steady state xxxxxxx approximation assumed that $-k_{\text{shed}DM_3}DM_3-k_{31DM}(DM_3) - k_{eDM3}DM_3+k_{\text{on}3}D_3 \cdot M_3 - k_{\text{off}3}(DM_3) = 0 $, we define $k_{ssd}=\frac{k_{\text{shed}DM_3} +k_{31DM}+ k_{eDM3}+ k_{\text{off}3}}{k_{\text{on}3}}$. 
These approximations are obtained from equation (16) at steady state, i.e. $\frac{d(DM_3)}{dt}=0$, and assuming the terms $ \frac{V_{DM_1}}{V_{DM_3}}k_{13DM}(DM_1)$ and $ k_{31DM}(DM_3) $ are negligible for the QSS. We assume that only $\frac{V_{DM_1}}{V_{DM_3}}k_{13DM}(DM_1)$ is negligible for the QSSD. 

We obtain the following approximations 
\begin{align*}
\text{AFIRT} \approx k_{ssd}\frac{M_{3\text{tot,ss}}}{M_{3,0}}\frac{\text{CL}\times\tau}{\text{B}\times\text{Dose}},\\
\text{AFIRT} \approx k_{ss}\frac{M_{3\text{tot,ss}}}{M_{3,0}}\frac{\text{CL}\times\tau}{\text{B}\times\text{Dose}},\\
\text{AFIRT} \approx k_{d}\frac{M_{3\text{tot,ss}}}{M_{3,0}}\frac{\text{CL}\times\tau}{\text{B}\times\text{Dose}},
\end{align*}  
where 
\begin{align*}
M_{3\text{tot,ss}}&=\frac{k_{13DM}(V_C/V_T)k_{\text{syn}M1}+(k_{eDM1}+k_{\text{shed}DM_1}+k_{13DM})k_{\text{syn}M3}}{(k_{eDM1}+k_{\text{shed}DM_1}+k_{13DM})(k_{eDM3}+k_{\text{shed}DM_3}+k_{31DM})-k_{31DM}k_{13DM}},\\
M_{3,0}&=\frac{k_{13M}(V_C/V_T)k_{\text{syn}M1}+(k_{eM1}+k_{\text{shed}M_1}+k_{13M})k_{\text{syn}M3}}{(k_{eM1}+k_{\text{shed}M_1}+k_{13M})(k_{eM3}+k_{\text{shed}M_3}+k_{31M})-k_{31M}k_{13M}},\\
\text{B}&=\frac{k_{13D}(V_C/V_T)}{k_{eD3}+k_{31D}}.
\end{align*}
We can notice from the picture below that the approximation obtained using the (QSS) hypothesis gives better results than the one using (QE).

% ---------------------------------------------------------------
% Soluble Target 
% ---------------------------------------------------------------
\section{Soluble target at Initial State and Steady State}
\subsection{Some formulas}
With the same notations as in the sections above.\newline 
\textbf{At initial state}, we have
\begin{align*}
    \frac{dS_1}{dt} & = 0 \\
    \frac{dS_3}{dt} & = 0 \\
                M_1 & = M_{10} \\
                M_3 & = M_{30} \\
                D_1 & = 0 \\
                D_3 & = 0 \\
               DS_1 & = 0 \\
               DS_3 & = 0 
\end{align*}
So ODE 13 and 14 give us the following linear system
\[
    \matrix{-(k_{13S} + k_{eS1})}{\frac{V_T}{V_C}k_{31S}}{\frac{V_C}{V_T}k_{13S}}{-(k_{31S} + k_{eS3})}
        \cdot \vector{S_1}{S_3}_0 = \vector{-k_{synS1} - k_{shedM1}M_{10}}{-k_{synS3} - k_{shedM3}M_{30}}.
\]
So the initial soluble target concentration in the central and tumor compartment is
\begin{align}
    S_{10} & = \frac{(k_{31S} + k_{eS3})\cdot(k_{synS1} + k_{shedM1}M_{10})
                    + \frac{V_T}{V_C}k_{31S}(k_{synS3} + k_{shedM3}M_{30})}
                {(k_{13S} + k_{eS1})(k_{31S} + k_{eS3}) - k_{13S}k_{31S}} \\
    S_{30} & = \frac{\frac{V_C}{V_T}\cdot k_{13S}\cdot(k_{synS1} + k_{shedM1}M_{10})
                    + (k_{13S} + k_{eS1})(k_{synS3} + k_{shedM3}M_{30})}
            {(k_{13S} + k_{eS1})(k_{31S} + k_{eS3}) - k_{13S}k_{31S}}.
\end{align}

\textbf{At steady state}, we assume that all soluble target are in the form of bounded complex.
By the symmetry of our model, we have
\begin{align*}
    S_{1tot,ss} & = \frac{(k_{31DS} + k_{eDS3})\cdot(k_{synS1} + k_{shedDM1}M_{1tot,ss})
                    + \frac{V_T}{V_C}k_{31DS}(k_{synS3} + k_{shedDM3}M_{3tot,ss})}
                {(k_{13DS} + k_{eDS1})(k_{31DS} + k_{eDS3}) - k_{13DS}k_{31DS}} \\
    S_{3tot,ss} & = \frac{\frac{V_C}{V_T} k_{13DS}\cdot(k_{synS1} + k_{shedDM1}M_{1tot,ss})
                    + (k_{13DS} + k_{eDS1})(k_{synS3} + k_{shedDM3}M_{3tot3,ss})}
            {(k_{13DS} + k_{eDS1})(k_{31DS} + k_{eDS3}) - k_{13DS}k_{31DS}}
\end{align*}

Formulas for computing membrane target at initial and steady state can be found in 
\emph{Model F Appendix}, for convenience, we include them below

\newcommand{\Det}{(k_{shedM1} + k_{13M} + k_{eM1})(k_{shedM3} + k_{31M} + k_{eM3}) - k_{13M}k_{31M}}

\newcommand{\DetSS}{(k_{shedDM1} + k_{13DM} + k_{eDM1})(k_{shedDM3} + k_{31DM} + k_{eDM3})
- k_{13DM}k_{31DM}}
\begin{align*}
    M_{10} & = \frac{(k_{shedM3} + k_{31M} + k_{eM3})k_{synM1} + \frac{V_T}{V_C}k_{31M}k_{synM3}}
                    {\Det} \\
    M_{30} & = \frac{\frac{V_C}{V_T}k_{13M}k_{synM1} + (k_{shedM1} + k_{13M} + k_{eM1})k_{synM1}}
                    {\Det} \\
    M_{1tot,ss} & = \frac{(k_{shedDM3} + k_{31DM} + k_{eDM3})k_{synM1} + 
                            \frac{V_T}{V_C}k_{31DM}k_{synM3}}
                    {\DetSS} \\
    M_{3tot,ss} & = \frac{\frac{V_C}{V_T}k_{13DM}k_{synM1} + 
                            (k_{shedDM1} + k_{13DM} + k_{eDM1})k_{synM1}}
                    {\DetSS}
\end{align*}


\subsection{Pseudo-Code for extending the core functions}
The formulas to compute various AFIRTS for soluble targets in the tumor compartment are 
\begin{align}
    \text{AFIRTS.Kssd} & = \text{Kssd.S}\times \frac{\text{Tacc.tum.S}}{\text{B}\times C_{avg1}} \\
    \text{AFIRTS.Kss}  & = \text{Kss.S}\times \frac{\text{Tacc.tum.S}}{\text{B}\times C_{avg1}} \\
    \text{AFIRTS.Kd}   & = \text{Kd.S}\times \frac{\text{Tacc.tum.S}}{\text{B}\times C_{avg1}} \\
\end{align}
Where $\text{Kssd.S}$, $\text{Kss.S}$ and $\text{Kd.S}$
\begin{align}
    \text{Kssd.S} & = \frac{k_{shedDM3} + k_{31DS} + k_{eDS} + k_{off3}}{k_{on3}} \\
    \text{Kss.S}  & = \frac{k_{shedDM3} + k_{eDS} + k_{off3}}{k_{on3}} \\
    \text{Kd.S}   & = \frac{k_{off3}}{k_{on3}} \\
\end{align}
 Finally,
\begin{align}
    \text{Tacc.tum.S} = \text{S3tot.ss} / \text{S30}
\end{align}


To compute AFIRT for soluable target from simulation, do
\begin{enumerate}
\item compute $\text{Sfree.pct}:=\frac{S_3}{S_{30}}$
\item take the average of Sfree.pct in the steady steady state
\item return the result from step 2
\end{enumerate}



% ---------------------------------------------------------------
% Similarity Transform
% ---------------------------------------------------------------

\section{Similarity Transform}

We consider the model in Figure 1. A drug $(D)$ binds to its target $(M)$ to form a complex $(DM)$. It has three compartments, central, tissue, and peripheral.

The drug and target dynamics are modeled with the following system of ordinary differential equations
\begin{align}
\frac{dD_1}{dt} &= \frac{1}{V_1}\text{Dose}_{iv}(t) - k_{12D}D_1 + \frac{V_3}{V_1}k_{21D}D_2 - k_{13D}D_1 \nonumber \\
&+ \frac{V_2}{V_1}k_{31D}D_3 - k_{\text{on}1}D_1 \cdot M_1 + k_{\text{off}1}(DM_1) - k_{eD1}D_1 \\
\frac{dD_2}{dt} &= k_{12D}\frac{V_1}{V_3}D_1 - k_{21D}D_2 \\
\frac{dD_3}{dt} &= \frac{V_1}{V_2}k_{13D}D_1 - k_{31D}D_3 - k_{\text{on}3}D_3 \cdot M_3 + k_{\text{off}3}(DM_3) - k_{eD3}D_3 \\
\frac{dM_1}{dt} &= k_{\text{syn}1} - k_{13M}M_1 + \frac{V_2}{V_1}k_{31M}M_3 - k_{\text{on}1}D_1 \cdot M_1 + k_{\text{off}1}(DM_1) \nonumber \\ 
&- k_{eM1}M_1 \\
\frac{dM_3}{dt} &= k_{\text{syn}3} + \frac{V_1}{V_2}k_{13M}M_1 - k_{31M}M_3 - k_{\text{on}3}D_3 \cdot M_3 + k_{\text{off}3}(DM_3) \nonumber \\ 
&- k_{eM3}M_3 \\
\frac{d(DM_1)}{dt} &= - k_{13DM}(DM_1) + \frac{V_2}{V_1}k_{31DM}(DM_3) + k_{\text{on}1}D_1 \cdot M_1 - k_{\text{off}1}(DM_1) \nonumber \\ 
&- k_{eDM1}(DM_1) \\
\frac{d(DM_3)}{dt} &= \frac{V_1}{V_2}k_{13DM}(DM_1) - k_{31DM}(DM_3) + k_{\text{on}3}D_3 \cdot M_3 - k_{\text{off}3}(DM_3) \nonumber \\ 
&- k_{eDM3}(DM_3) \\
\end{align}

We will calculate $D_{\text{tot,avg}3}$, which is the average drug concentration at steady state in the tissue. We have $D_{\text{tot,avg}3} = B \cdot D_{\text{tot,avg}1}$, where $D_{\text{tot,avg}1}$ is the average drug concentration at steady state in the central, and $B$ is the antibody biodistribution coefficient. Assume that drug elimination only occurs in the central compartment, i.e., $k_{eD3} = 0$. Then for drugs dosed with linear PK at regular intervals and concentration much larger than target concentration, we have $D_{\text{tot,avg}1} = \text{Dose}/(CL \cdot \tau)$, where $CL = k_{eD1} \cdot V_1$ is the drug clearance [CITE REFERENCE 25 FROM AFIR ARTICLE]. \\

For the model with drug elimination in both the central and tissue compartments, we will derive an alternative model with $k_{eD3} = 0$ that is indistinguishable from the existing model by using the similarity transform technique [CITE GODFREY 1989]. We consider drug concentration large enough that target binding does not affect drug distribution. Then the model equations for the drug can be written as 
\begin{align}
\frac{d\mathbf{D}}{dt} = A \cdot \mathbf{D}(t) + B \cdot Dose_{iv}(t), \label{model matrix}
\end{align}
with measurement
\begin{align}
\mathbf{m}(t) = C \cdot \mathbf{D}(t), \label{measurement matrix}
\end{align}
where
\begin{align}
\mathbf{D} = \begin{pmatrix}
D_1 \\
D_2 \\
D_3
\end{pmatrix},
\end{align}
\begin{align}
A = \begin{pmatrix}
-(k_{13D} + k_{12D} + k_{eD1}) & \dfrac{V_3}{V_1}k_{21D} & \dfrac{V_2}{V_1}k_{31D} \\
\dfrac{V_1}{V_3}k_{12D}        & -k_{21D}                & 0 \\
\dfrac{V_1}{V_2}k_{13D}        & 0                       & -(k_{31D} + k_{eD3})
\end{pmatrix},
\end{align}
\begin{align}
B = \begin{pmatrix}
\dfrac{1}{V_1} \\
0 \\
0
\end{pmatrix},
\end{align}
\begin{align}
C = \begin{pmatrix}
1 & 0 & 0 \\
0 & 0 & 1 
\end{pmatrix}.
\end{align}

We transform this model into an indistinguishable model with coefficient matrices $A'$, $B'$, and $C'$ such that $k'_{eD3} = 0$. This is accomplished with the similarity transform
\begin{align}
A' := TAT^{-1} && B' := TB && C' := CT^{-1}, \label{similarity transform}
\end{align}
for some transformation matrix
\begin{align}
T = \begin{pmatrix}
t_{11} & t_{12} & t_{13} \\
t_{21} & t_{22} & t_{23} \\
t_{31} & t_{32} & t_{33} 
\end{pmatrix} &&
T^{-1} = \begin{pmatrix}
\hat{t}_{11} & \hat{t}_{12} & \hat{t}_{13} \\
\hat{t}_{21} & \hat{t}_{22} & \hat{t}_{23} \\
\hat{t}_{31} & \hat{t}_{32} & \hat{t}_{33} 
\end{pmatrix}.
\end{align}
Alternatively, one can get this similarity transform with the change of variables $\mathbf{D} = T^{-1}\mathbf{D'}$. Then from (\ref{model matrix}) and (\ref{measurement matrix}) we have
\begin{align}
\frac{d(T^{-1}\mathbf{D'})}{dt} = A \cdot (T^{-1}\mathbf{D'})(t) + B \cdot Dose_{iv}(t), \qquad \mathbf{m}(t) = C \cdot (T^{-1}\mathbf{D'})(t). \label{change of variables}
\end{align}
Multiplying the first equation in (\ref{change of variables}) on the left by $T$ yields
\begin{align}
\frac{d\mathbf{D'}}{dt} = TAT^{-1} \cdot \mathbf{D'}(t) + TB \cdot Dose_{iv}(t).
\end{align}
Then we have the new model equation and measurement
\begin{align}
\frac{d\mathbf{D'}}{dt} = A' \cdot \mathbf{D'}(t) + B' \cdot Dose_{iv}(t), \qquad \mathbf{m}(t) = C' \cdot \mathbf{D'}(t),
\end{align}
where $A'$, $B'$, and $C'$ are given by (\ref{similarity transform}).

Regardless of model formulation, the same dose is given to $D_1$, that is,
\begin{align}
B' = B.
\end{align}
From this and (\ref{similarity transform}) we get $t_{11} = 1$, $t_{21} = 0$, and $t_{31} = 0$. Regardless of model formulation, we measure $D_1$ and $D_3$, that is,
\begin{align}
C' = C.
\end{align}
From this and (\ref{similarity transform}) we get $\hat{t}_{11} = 1$, $\hat{t}_{12} = 0$, $\hat{t}_{13} = 0$, $\hat{t}_{31} = 0$, $\hat{t}_{32} = 0$, and $\hat{t}_{33} = 1$. For a $3 \times 3$ matrix, the inverse is given by
\begin{align}
T^{-1} = \dfrac{1}{\det{T}}\begin{pmatrix}
t_{22}t_{33} - t_{23}t_{32} & t_{13}t_{32} - t_{12}t_{33} & t_{12}t_{23} - t_{13}t_{22} \\
t_{23}t_{31} - t_{21}t_{33} & t_{11}t_{33} - t_{13}t_{31} & t_{13}t_{21} - t_{11}t_{23} \\
t_{21}t_{32} - t_{22}t_{31} & t_{12}t_{31} - t_{11}t_{32} & t_{11}t_{22} - t_{12}t_{21} 
\end{pmatrix},
\end{align}
with the determinant given by 
\begin{align}
\det{T} = t_{11}(t_{22}t_{33} - t_{23}t_{32}) - t_{12}(t_{21}t_{33} - t_{23}t_{31}) + t_{13}(t_{21}t_{32} - t_{22}t_{31}).
\end{align}
Putting the above findings into $T^{-1}$ yields $t_{12} = 0$, $t_{13} = 0$, $t_{32} = 0$, and $t_{33} = 1$. From $TT^{-1} = I$, we get $t_{22} = 1$. Then we have
\begin{align}
T = \begin{pmatrix}
1 & 0 & 0 \\
0 & 1 & t_{23} \\
0 & 0 &  1
\end{pmatrix} &&
T^{-1} = \begin{pmatrix}
1 & 0 & 0 \\
0 & 1 & -t_{23} \\
0 & 0 &  1
\end{pmatrix}.
\end{align}
Putting this into (\ref{similarity transform}) yields
\begin{align}
A' = \begin{pmatrix}
-(k_{13D} + k_{12D} + k_{eD1})                           & \dfrac{V_3}{V_1}k_{21D} & -\dfrac{V_3}{V_1}k_{21D}t_{23} + \dfrac{V_2}{V_1}k_{31D} \\
\dfrac{V_1}{V_3}k_{12D} + \dfrac{V_1}{V_2}k_{13D}t_{23}  & -k_{21D}                & (k_{21D} - k_{31D} - k_{eD3})t_{23}                      \\
\dfrac{V_1}{V_2}k_{13D}                                  & 0                       & -(k_{31D} + k_{eD3})
\end{pmatrix}.
\end{align}
Now we impose 
\begin{align}
k'_{eD3} := -a'_{33} - \dfrac{V_1}{V_2}a'_{13} = 0,
\end{align}
from which we get $t_{23} = -(V_2/V_3)(k_{eD3}/k_{21D})$. After putting this last piece into $A'$, we get $k'_{eD1}$ from
\begin{align}
k'_{eD1} &= -a'_{11} - \dfrac{V_3}{V_1}a'_{21} - \dfrac{V_2}{V_1}a'_{31} \nonumber \\ 
&= k_{eD1} - \dfrac{k_{13D} \cdot k_{eD3}}{k_{21D}}. \label{new k'_{eD1}}
\end{align}
Thus $CL' = k'_{eD1} \cdot V_1$, with $k'_{eD1}$ given by (\ref{new k'_{eD1}}). And finally we get $D_{\text{tot,avg}1} = \text{Dose}/(CL' \cdot \tau)$, and $D_{\text{tot,avg}3} = B \cdot D_{\text{tot,avg}1}$. \\

\textit{Remark}: In the similarity transform, we measure $D_1$ and $D_3$, which is indicated by $C$. As a result of imposing $C' = C$, the original model and the transformed model output the same values for these variables. $D'_2$, however, differs from $D_2$.