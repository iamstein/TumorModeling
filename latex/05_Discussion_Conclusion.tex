\section{Discussion}

The key insight from this work is that under many clinically relevant scenarios, AFIRT can be estimated using four parameters: a binding constant ($\Keq$), the target accumulation ratio ($\Tacc$), the average drug concentration in circulation ($\Cavg$) and the biodistribution coefficient for the drug to the tissue of interest ($B$)
\begin{equation}
	\AFIRT = \frac{\Keq \cdot \Tacc}{B \cdot \Cavg}
\end{equation}
The binding constant ($\Keq$) above depends on the nature of the target of interest.
\begin{itemize}
	\item For soluble targets, a surface plasmon resonance estimate for $\Kd$ may be sufficient.  It should be noted that that there have been scenarios where the estimate for the in vitro and in vivo $\Kd$ differed by 1000-fold (e.g. TNF-$\alpha$ \cite[Figure 8]{meno05}).
	\item For membrane-bound targets on cells where trafficking between tissue and circulation is negligible, $\Kss$ can be estimated by a cell-based assay.
	\item For membrane-bound targets on cells where trafficking between tissue and circulation is rapid, $\Kssd$ may be needed.  As it is not clear how to directly estimate this parameter, one solution is to adjust a cell-based estimate for $\Kss$ by a "safety factor."
\end{itemize}


\begin{itemize}
\item Key Insight = AFIRT
	\begin{itemize}
		\item Just need a few terms, don't need a big mechanistic model
		\item Mention affect of doubling dose, halving $\Kss$ etc.
			\subitem But halving $\Kss$ is different from halving $\Kd$.
	\end{itemize}
\item Review Assumptions needed for AFIRT to hold
	\begin{itemize}
		\item large drug concentration
		\item distribution of drug-target complex out of tissue can be ignored
		\item fluctuations during dosing interval can be ignored
		\item others?
	\end{itemize}
\item Importance of using $\Kss$ or even $\Kssd$ from cell based assays instead of $\Kd$.  The more realistic the preclinical system, the better.
\item Things that are ignored
	\begin{itemize}
		\item Feedback, though this would be easy to include.  Just need to have different ksyn for initial condition and steady state.
		\item Low-Intermediate concentrations, especially relevant for ADCs and bispecifics.
		\item Competition for target binding sites between drug and endogenous ligand.
	\end{itemize}
\item Applications
	\begin{itemize}
		\item Can rapidly assess new drugs
		\item Can help inform dose finding
	\end{itemize}
\end{itemize}

Also mention connection with L50 from \cite{gabrielsson17}

\section{Conclusion}